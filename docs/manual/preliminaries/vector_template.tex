\index{Vector class}

The \lstinline"Vector" class is a template, which means that it
can be applied to different types \cite{Eckel2000}. That is, we
can create a \lstinline"Vector" or \lstinline"int" numbers, \lstinline"MyClass" objects, etc.

The \lstinline"Vector" in \texttt{OpenNN} is derived from the \lstinline"vector" in the Standard Template Library. 

\subsubsection*{Members}

The onl member of the \lstinline"Vector" class is:

\begin{itemize}
\item[-] A double pointer to some type. 
\end{itemize}

That two class members are declared as being private. 

\subsubsection*{File format}

Vector objects can be serialized or deserialized to or from a data file which contains the member values. The file format is as follows.

\begin{lstlisting}
element_0 element_1 ... element_N
\end{lstlisting}

\subsubsection*{Constructors}

Multiple constructors are defined in the \lstinline"Vector" class, where the different constructors take different parameters. 

% Default constructor

The easiest way of creating a vector object is by means of the default constructor, wich builds a vector of size zero.
For example, in order to construct an empty \lstinline"Vector" of
\lstinline"int" numbers we use

\begin{lstlisting}
Vector<int> v;
\end{lstlisting}

% Size constructor

The following sentence constructs a \lstinline"Vector" of $3$
\lstinline"double" numbers.

\begin{lstlisting}
Vector<double> v(3);
\end{lstlisting}

% Size initialization constructor

If we want to construct \lstinline"Vector" of $5$ \lstinline"bool"
variables and initialize all the elements to $false$, we can use

\begin{lstlisting}
Vector<bool> v(5, false);
\end{lstlisting}

% File constructor

It is also possible to construct an object of the \lstinline"Vector" class and at the same time 
load its members from a file. In order to do that we can do

\begin{lstlisting}
Vector<int> v(`Vector.dat');
\end{lstlisting}

The file `Vector.dat' contains a first row with the size of the vector and an aditional row for each element of the vector. 

% Copy constructor

The following sentence constructs a \lstinline"Vector" which is a copy of another \lstinline"Vector",

\begin{lstlisting}
Vector<MyClass> v(3);
Vector<MyClass> w(v);
\end{lstlisting}

\subsubsection*{Operators}

The \lstinline"Vector" class also implements different types of operators for assignment, reference, arithmetics or comparison. 

% Assignment operator 

The assignment operator copies a vector into another vector, 

\begin{lstlisting}
Vector<int> v;
Vector<int> w = v;
\end{lstlisting}

% Reference operator 

The following sentence constructs a vector and sets the values of their elements using the reference operator. Note
that indexing goes from $0$ to $n-1$, where $n$ is the
\lstinline"Vector" size.

\begin{lstlisting}
Vector<double> v(3);
v[0] = 1.0;
v[1] = 2.0;
v[2] = 3.0;
\end{lstlisting}

% Arithmetic operators 

Sum, difference, product and quotient operators are included in the \lstinline"Vector" class to perform arithmetic operations with a scalar or another \lstinline"Vector". Note that the arithmetic operators with another \lstinline"Vector" require that they have the same sizes. 

The following sentence uses the vector-scalar sum operator, 

\begin{lstlisting}
Vector<int> v(3, 1.0);
Vector<int> w = v + 3.1415926;
\end{lstlisting}

An example of the use of the vector-vector multiplication operator is given below, 

\begin{lstlisting}
Vector<double> v(3, 1.2);
Vector<double> w(3, 3.4);
Vector<double> x = v*w;
\end{lstlisting}

% Arithmetic and assignment operators

Assignment by sum, difference, product or quotient with a scalar or another \lstinline"Vector" is also possible by using the arithmetic and assignent operators. If another \lstinline"Vector" is to be used, it must have the same size.

For instance, to assign by difference with a scalar, we migh do

\begin{lstlisting}
Vector<int> v(3, 2);
v -= 1;
\end{lstlisting}
 
In order to assign by quotation with another \lstinline"Vector", we can write 

\begin{lstlisting}
Vector<double> v(3, 2.0);
Vector<double> w(3, 0.5);
v /= w;
\end{lstlisting}

% Equality and relational operators

Equality and relational operators are also implemented here. They can be used with a scalar or another \lstinline"Vector". For the last case the same sizes are assumed. 

An example of the equal to operator with a scalar is 

\begin{lstlisting}
Vector<bool> v(5, false);
bool is_equal = (v == false);
\end{lstlisting}

The less than operator with another \lstinline"Vector" can be used as follows,

\begin{lstlisting}
Vector<int> v(5, 2.3);
Vector<int> w(5, 3.2);
bool is_less = (v < w);
\end{lstlisting}

\subsubsection*{Methods}

Get and set methods for each member of this class are implemented to exchange information among objects. 

% Size and element methods

The method \lstinline"size" returns the size of a \lstinline"Vector".

\begin{lstlisting}
Vector<MyClass> v(3);
int size = v.size();
\end{lstlisting}

On the other hand, the method \lstinline"set" sets a new size to a \lstinline"Vector". 
Note that the element values of that \lstinline"Vector" are lost. 

\begin{lstlisting}
Vector<bool> v(3);
v.set(6);
\end{lstlisting}

% Initialization methods

If we want to initialize a vector at random we can use the \lstinline"initialize_uniform" or \lstinline"initialize_normal" methods, 

\begin{lstlisting}
Vector<double> v(5);
v.initialize_uniform();
Vector<double> w(3);
w.initialize_normal();
\end{lstlisting}

% Mathematical methods

The \lstinline"Vector" class also includes some mathematical methods which can be useful in the development of neural networks algorithms and applications. 

The \lstinline"calculate_norm" method calculates the norm of the vector, 

\begin{lstlisting}
Vector<double> v(5, 3.1415927);
double norm = v.calculate_norm();
\end{lstlisting}

In order to calculate the dot product between this \lstinline"Vector" and another \lstinline"Vector" we can do

\begin{lstlisting}
Vector<double> v(3, 2.0);
Vector<double> w(3, 5.0);
double dot = v.dot(w);
\end{lstlisting}

We can calculate the mean or the standard deviation values of the elements in a \lstinline"Vector" by using the \lstinline"calculate_mean" and \lstinline"calculate_standard_deviation" methods, respectively. For instance

\begin{lstlisting}
Vector<double> v(3, 4.0);
double mean = v.calculate_mean();
double standard_deviation = v.calculate_standard_deviation();
\end{lstlisting}

% Utility methods

Finally, utility methods for serialization or loading and saving the class members to a file are also included. 
In order to obtain a \lstinline"std::string" representation of a \lstinline"Vector" object we can make
 
\begin{lstlisting}
Vector<bool> v(1, false);
std::string vector_string = v.to_string();
\end{lstlisting}

To save a \lstinline"Vector" object to a file we can do

\begin{lstlisting}
Vector<int> v(2, 0);
v.save(`Vector.dat');
\end{lstlisting}

The first row of the file \lstinline"Vector.dat" is the size of the vector and the other rows contain the values of the elements of that vector. 

If we want to load a \lstinline"Vector" object from a data file we could write

\begin{lstlisting}
Vector<double> v;
v.load(`Vector.dat');
\end{lstlisting}

Where the format of the \lstinline"Vector.dat" file must be the same as that described above. 


