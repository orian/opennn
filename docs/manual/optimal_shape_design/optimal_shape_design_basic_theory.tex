Optimal shape design is a very interesting field both mathematically
and for industrial applications. The goal here is to computerize the
design process and therefore shorten the time it takes to design or
improve some existing design. In an optimal shape design process one
wishes to optimize a criteria involving the solution of some
mathematical model with respect to its domain of definition
\cite{Mohammadi2004}. The detailed study of this subject is at the
interface of variational calculus and numerical analysis.

%For instance the design of a harbor which minimizes the waves
%coming from far can be done at little cost by standard
%optimization methods once the numerical simulation of Helmholtz
%equation is mastered.
%
%Weight reduction in car engine, aircraft structures, etc
%Electromagnetically optimal shapes, such as in stealth airplanes
%Wave canceling fore bulbe in boat design Drag reduction for
%airplanes,cars and boats. An example of an optimal shape design
%problem could be the design of flow aerofoil sections.

%Optimal Shape Design is concerned with the optimization of some
%performance criterion dependent (besides the constraints of the
%problem) on the physical form of some region (device).

%Optimal Shape Design has developed over the years, from the
%abstract field of calculus of variations to applications in
%structure mechanics and fluid mechanics becoming now a valuable
%tool for the design optimization of airplanes. Weight reduction,
%stress reinforcement, drag reduction and even noise reduction can
%be obtained. The field is said to have started with Hadamard[8]
%who gave the first formula to evaluate the change due to a
%boundary modification of the domain of a partial differential
%equation.

In order to properly define an optimal shape design problem the
following concepts are needed:

\begin{enumerate}
\item Mathematical model.
\item Neural network.
\item Performance functional. 
\item Training strategy.
\end{enumerate}

\subsection*{Mathematical model}
\index{mathematical model}
\index{shape variable}
\index{state variable}
\index{state equation}

The mathematical model or state equation is a well-formed formula which involves the
physical form of the device to be optimized. 
It contains shape variables and state variables. 

A mathematical model might be described by algebraic equations, ordinary differential equations or partial differential equations. 

\subsection*{Neural network}
\index{shape constraint}
\index{boundary condition}
\index{lower and upper bounds} 

A neural network is used to represent the shape variables. 
Optimal shape design problems are usually defined by constraints on the shape function. 
Two important types of shape constraints are boundary conditions and lower and upper bounds. 

\subsection*{Performance functional}
\index{state constraint}
\index{admissible shape}
\index{admissible state}

The performance functional of an optimal shape design problem always includes an objective term.
It usually includes a constraints term,  

\begin{eqnarray}\nonumber
\text{Performance functional = objective term + constraints term}. 
\end{eqnarray}


An optimal shape design problem might also be specified by a set of
constraints on the state variables of the
device.

State constraints are conditions that the solution to the problem
must satisfy. This type of constraints vary according to the problem
at hand.

In this way, a design which satisfies all the shape and state
constraints is called an admissible shape.

Similarly, a state which satisfies the constraints is called an
admissible state.

The performance criterion expresses how well a given design does the
activity for which it has been built.


Optimal shape design problems solved in practice are, as a rule,
multi-criterion problems. This property is
typical when optimizing the device as a whole, considering, for
example, weight, operational reliability, costs, etc. It would be
desirable to create a device that has extreme values for each of
these properties. However, by virtue of contradictory of separate
criteria, it is impossible to create devices for which each of them
equals its extreme value.


\subsection*{Trainining strategy}

The performance functional for optimal shape design problems might contain up to three terms: 
objective, regularization and constraints. 
On the other hand, in most of the cases, it cannot be computed analitycally. 
That makes that a single training algorithm might not fully converge if the solution is far away from the optimal one. 

Therefore, when solving optimal shape design problems, it is recommended to use an initialization training algorithm before the main training process. 
The form of the training strategy is therefore as follows:

\begin{eqnarray}\nonumber
\text{Training strategy: initialization training algorithm, main training algorithm}. 
\end{eqnarray}

The initialization training algorithm is usually a zero order algorithm, such as random search or the evolutionary algorithm;
the main training algorithm might be a first order algorithm, such as conjugate gradient or the quasi-Newton method. 
